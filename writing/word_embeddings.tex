Consider the common scenario of a text classifier which maps plain text files to one of several classes.  It is common for the plain text to first be processed into a sequence of tokens, which are then each assigned an integer resulting in a sequence of integers.

\begin{definition}
Let $s$ be a sequence of characters.  Let $a_n \in \{0,1,\dots,V\} \forall n \in \{0,1,\dots,N\}$ and $E:s\rightarrow \{a_n\}_{n=1}^{N_s}$.  Then $E$ is called an encoder, $V$ the encoder vocabulary size, and $N_s$ the sample length with respect to $E$.
\end{definition}

In plain words, an encoder maps a string to a finite sequence of bounded integers.  The sequence length depends on both the encoder and the string.  Assume a fixed encoder, and therefore vocabulary size, $V$.  Since, after encoding, the distance between one word and another is arbitrary, the numerical representation can be further translate into a one-hot encoded vector.  That is, the integer $n$ is mapped to a vector where the $n^{th}$ element is $1$ and all others are $0$.  This ensures that all vectors representing words are unit norm and the distance between any two different words is the same.  The set of one-hot encoded vectors of size $V$ is denoted as $1_V$.

This simple method of representing words as vectors results in a very high dimension representation of all words in the vocabulary, and thus even a very simple linear model would be very large and difficult to train.  Using the word2vec model discussed in the previous chapter, the dimension of this representation can be significantly reduced, while also encoding information about statistical semantic similarity about each word.

\begin{definition}
Let $f: 1_V \rightarrow \mathbb{R}^D$.  the function $f$ is called a word embedding and $D$ is the size of $f$, or embedding size.
\end{definition}

\noindent
Let $W \in M_{D\times V}(\mathbb{R})$.  Then clearly any word embedding, $f$, of size $D$ may be represented as the matrix multiplication $Wv$ $\forall v \in 1_V$.  The matrix $W$ is called the embedding matrix.  This numerical representation of words is extremely useful since it allows one to apply more general and modern techniques to solving the problem of classification.
